\newcommand{\nathanln}{???????}

\section{Computer Science in Lee County}\label{sec:CSInLee}

TEALS began as the brainchild of Kevin Wang, who, while an employee of Microsoft
in Redmond, Washington, began teaching a course at ????????\footnote{For more
details on the history of TEALS, please see Section~\ref{sec:TEALS}.} When he
was just about to leave his job to teach full-time Microsoft recognized the
great benefit in the model that Mr.~Wang had set up, decided instead that they
would fund the organization's expansion. TEALS began as a project in the Seattle
region alone, but, at the urging of Audrey Sniezek, opened an experimental
project at Lee County High School (LCHS) in Beattyville, Kentucky.

In that first year, Isaac Wilson, Nathan \nathanln{}, and Ms.~Sniezek taught an
introductory course on computer science to six students at Lee County High
School. This was made possible through the generous donations of Microsoft
to beef up the computer technology in the high school to support the
set up as well as by the recent efforts of AT\&T to expand broadband
access in the region\cite{Something in which Hal Rogers is talking about Silicon
Holler}. Throughout the year, Messrs.~Wilson and \nathanln{} taught the course
remotely and Ms.~Sniezek made frequent appearances in the classroom during her
sojourns to climb the beautiful and challenging sandstone rock faces in the
region.

The results were astounding. ????????????? Some stuff about what people are
doing now. Moreover, there was strong interest from the two students who had
not graduated to continue their computer science education. As such, TEALS and
LCHS decided to expand the program to include many more students in the
introductory class and to begin an Advanced Placement course\footnote{See
Section~\ref{sec:ap} for an overview of the Adavanced Placement curriculum.}.
This course is currently being taught by Dan Goldin and Gabe Ghearing with
occassional help from Ms.~Sniezek.

Currently, in each of these classes Danny Wright, a long-time teacher at Lee
County, has been serving as the professional educator in charge. While the
volunteer teachers tend to have many years of professional experience
(Mr.~Ghearing runs a ????????? company and Mr.~Barnett has worked in both
Louisville and New York City as a software developer for over ????? years)
and several have a good deal of teaching experience (Dr.~K.~Wilson has spent
many years honing his teaching craft at universities, and Mr.~Bethune was an
English teacher for several years in Japan and the U.S.~before moving into
software engineering), Mr.~Wright gives insight into how the students with whom
he is extremely familiar as well as giving feedback on the experience of the
students in the classroom. He is also required to learn the material along with
his students so that he may take over the class in two years' time\footnote{
Though, due to personal issues, Mr.~Wright will not be the teacher in charge
of the class next year, but that teacher will have the same requirement to
learn the material along with the students and in fact is already beginning to
attend class to get a head start.}.

This setup seems to work very well at the introductory level: Many kids'
interets remain piqued, there is enough staff to handle the reduced face-time
that volunteering brings, and the curriculum is fluid enough that, at their
discretion, the lead volunteer teachers and Mr.~Wright may change trajectories
in accordance with the needs and capacities of their students.

However, with the introductory of the AP Computer Science course and its
rigorous, inflexible schedule, we have found that having a teacher on-site
at least once a week makes an enormous difference to the confidence and
competence of the students' in the course. Originally, Ms.~Sniezek performed
this role, spending many days after school with the AP students, especially
the girls in the course who seemed to be struggling the most. But, due to
changing employment requirements, Ms.~Sniezek has had to leave Lee County for
the moment. It was after she left that we noticed a sharp downturn in the
performance of several of the students in the AP class.

Twenty-six of the three hundred students at Lee County High School are currently
enrolled in Computer Science courses, so the interest is high. We are deeply
commited in continuing the program, but without an on-site computer science
expert, we are afraid that the AP Computer Science class cannot be maintained.
This proposal seeks to bring in a student expert from Eastern Kentucky
University on a semiweekly basis as well as form the foundation of a potential
long-term partnership between the University and the computer science education
community in Eastern Kentucky.
