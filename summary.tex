
\section{Executive Summary}

Technology Education And Literacy in Schools (TEALS) is a national non-profit organization started at and
mostly funded by Microsoft whose purpose is to bring Computer Science education
to more people around America \autocite{TEALSWeb}. As Computer Science skills are extraordinarily
useful both in terms of teaching logical reasoning\autocite{????} and in terms of
acquiring employment\footnote{For instance, J.~Wing, head of Computer Science at Carnegie Mellon has
said, ``Computation thinking is a fundamental skill for everyone, not just computer scientists.... Computational thinking involves solving problems, designing systems, and understanding human behavior, by drawing on the concepts fundamental to computer science. Computational thinking involves a range of mental tools that reflect the breadth of the field of computer science \autocite[pg.~33]{ComputationalThinking}.'' The Bureau of Labor Statistics predicts that jobs for software developers will increase 22\% in the decade from 2012--2022 among many other phenomenal growth projections in related fields \cite{BLS}. See also Section~\ref{sec:benefits}.} However,
expertise in Computer Science is an uncommon skill among professional educators\footnote{See Section~\ref{sec:CSInUSA}}.
TEALS seeks to reach its goals by enlisting community volunteers to teach
the classes' content while professional educators simultaneously advise the
experts on teaching strategies, assist in classroom management tasks, and
study the topic under the tutelage of the experts. The goal is that within two
years, the educators will have learned the material sufficiently well to take
over the classroom on their own\autocite{TEALSWeb}. This approach has had great
success in the Seattle area\autocite{Statistics?}, and
so the organization has spread to include schools in 11 states and the District
of Columbia \autocite{TEALSRoster}.

As an experiment, TEALS also expanded to include several rural regions. In these
cases, a dearth of community expertise has led TEALS to employ teleconferencing
technology and volunteers from the urban areas where such expertise is more
plentiful to serve as the outside experts. The first area in which this
particular approach was employeed was in Beattyville, at Lee County High School
in Kentucky, but the approach has since been tried in Duluth, Minnesota;
Anchorage, Alaska; and a few other regions of the United States\autocite{TEALSRoster}.

While the approach has seen relatively good success in introductory courses
\autocite{WoolyWormVideo}, it has seen relatively poor success in expanding to
the Advanced Placement (AP) courses in Lee County. We hypothesize that the
exacting nature of the AP curriculum\footnote{See Section\ref{sec:ap}.}
requires a certain amount of availability
of expertise in the topic to be constantly available. Given the remote nature of
the course and the fact that the professional educator is still learning the
material, on-site expertise is impossible, and given the voluntary nature of the
experts teaching the course, remote availability is limited. Moreover, even if
some remote method of providing help were possible, commensurate with the
poverty of Lee County, some of the students in the course do not even have an
internet connection at home with which they could access such help.

This grant proposes a partnership between the Lee County TEALS program and
the Computer Science Department of Eastern Kentucky University (EKU) under the
Center for Appalachian Regional Engagement and Stewardship
(CARES) Community Grants program. The budget would fund an on-site tutor who
would be a student at EKU studying computer science and who had proven mastery
of the skills taught in AP Computer Science. This tutor would be available for
tutoring at Lee County High School two days a week, would attend class once a
week, and would join the volunteer experts and the professional educator on an
online forum to help answer questions from students for two hours a week.
The tutor would also be required to attend five hours of training on the
curriculum with the experts and educator before the school year began. Including
travel expenses and a USD 20 per hour pay rate, we budget USD 4000 for the
tutor over the course of the entire 2014--2015 school year.

We view this as an amazing opportunity for all parties involved. For the tutor,
a chance to teach the skills that she has developed in classes as well as a
chance to get advice on computer science from industry professionals as well as
on teaching from a practicing educator. For the industry professionals, an
on-site semi-expert partner who is obligated by contract (instead of by honor) to
assist in their students' learning. For the professional educator, a teaching
assistant to help guide students they might otherwise feel powerless to help.
For EKU, a chance to participate in an experiment bringing computer science
education to a region that otherwise would not have any access to it. For
Kentucky, a chance to invigorate its education system and be on the forefront
of brining timely and interesting electives\footnote{Though, recent action in the General Assembly might allow students to take computer science as an alternative to foreign language as a requirement for graduation \autocite{CJSB16}} to the state
which will help power its economy and train its future leaders in the skills
most necessary for the moment. Moreover, if this partnership goes well, we
can imagine a similar tutor covering other schools in the service region of the
CARES program, potentially even as a new form of (pre-)student teaching in the
ever-innovating teacher training program at Eastern Kentucky\footnote{Is this
just bullshit?}.

Computer Science is a more and more desirable skill which exercises the mind
in ways similar to abstract science and mathematics education. But expertise is
still held among a very elite few. EKU through CARES has the opportunity to
expand access to that knowledge that can..........
