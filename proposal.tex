\section{Detailed Proposal}\label{sec:proposal}

In this section, we detail our proposal for a CARES Community Grant to help
fund a semi-weekly on-site tutor for the AP Computer Science course at
Lee County High School.

\subsection{Tutor requirements}\label{ssec:tutor}

The problem we are trying to solve is the need for an on-site expert. As such,
any potential tutor must be skilled object-oriented programming (at the level
of an A in CSC 190 and 191 at EKU) as well as having specific experience in
the Java programming language.

As the tutor will be serving as an auxiliary teacher, previous teaching or
tutoring experience is a plus, though a recommendation from a professor
about potential will be acceptable.

The tutor will be expected to attend two to five hours of training sessions
on the curriculum before the school year begins with the remote volunteers
and the in-class teacher.

The tutor will be expected to be at LCHS two times a week as long as school is
in session. Each visit to the school should last between one and two hours
depending on student interest and should be held either before or after school.
Moreover, if the AP Computer Science course can be scheduled as the first or
last period of the school day, then one of these four hours should be spent
in class.

Further, the tutor will be expected to spend approximately one hour of the week
answering student questions remotely and catching up with the in-class teacher
and the remote volunteers so that she may prepare for the coming week.

Thus, in total including driving time, this works out to 9--10 hours per week
that the tutor can expect to be working for the course.

\subsection{Organizational Chart}\label{ssec:orgchart}

The tutor will be hired by ????????? at Eastern Kentucky University in
consultation with Ms.~Audrey Sniezek of Microsoft and Dr.~Kevin Wilson
of Knewton. Once hired, Mr.~??????, the ???????? of ?????????, will be in charge
of tracking hours, managing dispersals of funds, and requesting reimbursements
from CARES. The tutor will be considered an independent contractor with no
employment ties to the University or the school.

\subsection{Budget}\label{ssec:budget}

The tutor would be paid USD 20 per hour of actual instruction and meeting time
and reimbursed at a rate USD 0.56 per mile, the standard rate of the Internal
Revenue Service. With 36 weeks of school and the five hours of training ahead
of time, the total reimbursement to be paid to the tutor would amount to
approximately USD 4000\footnote{Mileage estimates based on Google Maps
directions from Eastern Kentucky University to Lee County High School, an
estimated 42.6 miles. For reference, {\tt http://goo.gl/maps/C41bW}.}

As per the matching requirements of the grant, we propose that the work of the
volunteer teachers should count toward that total. The total number of hours
volunteered during the course of the school year is approximately 800 hours
by the full teaching staff. As a reference for the dollar cost of that labor,
Microsoft matches volunteer hours at a rate of
USD 17 per hour for employees\footnote{Note that only three of eight of the
volunteers at Lee County this year are Microsoft employees, and most of their
employers do not match volunteer hours with funds. The Microsoft rate is given
only as a reference to price out the labor.} \autocite{MicrosoftGuidebook}, and
so at that rate, approximately USD 13,600
in matching funds are being given by volunteers. Of that, USD 4,500 is directly
compensated as part of the agreement TEALS and LCHS have, leaving USD 9,100 in
available matching. As Lee County is a distressed county according to the
Appalachian Regional Commission \autocite{DistressedCounties}, only USD 2000 in
matching funds must be accounted for. This total is handily met.
