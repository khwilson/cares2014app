
\section{The Rigors of the Advanced Placement Curriculum}\label{sec:ap}

The Advanced Placement (AP) program was instituted by the College Board in ????.
Its purpose was to rigorize and standardize some high school curricula to
improve educational quality in the United States. Today it also serves as a
credentialing mechanism through which rising college freshmen can prove their
mastery of university-level material by standardized exams. Attaining a high
enough score on these exams often leads to college credit\cite{AP History}.
For instance, at Eastern Kentucky University, earning a score of 3 out of 5
on the AP Computer Science exam will garner ???? credits\footnote{To clarify
slightly, there are two different AP Computer Science exams, one of which covers
a superset of the material covered in the other. Scoring 3 or higher on the
broader exam garners ????? credits at EKU. However, the narrower curriculum is
being taught at Lee County.}.

Advanced Placement curricula are very rigorous, often much more so than the
standard curriculum required by the various state boards of education. Moreover,
a student's mastery of the material is measured by a comprehensive examination
at the end of the year. One of the great challenges of teaching the AP
curriculum is that this exam occurs each year in early May, irrespective of the
schedule of the various schools teaching the material. As such, teachers of AP
courses are necessarily much less flexible in their timelines and students are
under much stronger pressures to keep up. Falling behind, even by a few days,
can be extremely dangerous: either the teacher must fall behind on the brutal
schedule, or the student may irreparably fall behind learning the material. This
problem is especially pronounced in more quantitative curricula where
prerequisite structures are strong as opposed to, say, U.S.~History where
not understanding the causes and effects of the Seven Years' War will not
doom your understanding of the Civil War.

The Computer Science A AP curriculum is especially dense. Students are expected
to have read and understand to the point of actionable insights a ????? page
book on the Java computer programming language \cite{the java book, the ap
curriculum}. On top of this, the students are expected to completed several
projects, including one particular standardized project on which they will be
examined at the end of the semester. Each of these projects can take upwards of
100 hours of work. And as is evinced by the credits EKU gives for successfully
completing this course, the amount of efforts expected to be expended is in
line with a ????-credit college course.

Students who successfully complete the AP Computer Science A curriculum are
expected to succeed in advanced computer science classes at a university at
the level typically expected of a college sophomore of junior. As such, it is
expected that some students will struggle with the material. But two of the
hardest struggles for many of these students will be the standard problems
of college freshman: time management and access to expertise. We are not out
to solve the former problem as that is a problem faced by all sufficiently
advanced high schoolers.

But the latter problem, access to expertise, is one that we can solve. In
college it is solved by having a very large cohort of students struggling
through the same material so that co-learners can support each other and
by making professors and graduate students available for individual or small
group consultation in office hours. For instance, at computer science powerhouse
Carnegie Mellon University, office hours for the introductory computer science
course (the equivalent of the AP Computer Science curriculum) span ???? of the
60 hours between 8am and 8pm Monday through Friday \cite{SomeCMUWebsite}.
Moreover, approximately ???? people took this course in the Fall Semester of
2013\footnote{These estimates are based on public data available representing
the number of spots open in the course availble at \cite{CMUDirectory}.}.

Compartively, six students are taking the inaugural AP Computer Science course
at Lee County High School and experts are only available for one hour a day each
weekday during class. This disparity makes success enormously difficult.
